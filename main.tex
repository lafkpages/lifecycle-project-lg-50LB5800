\documentclass[12pt, letterpaper]{article}
%\documentclass[10pt, letterpaper, twocolumn]{article}

% Images
\usepackage{graphicx}
\graphicspath{{images}}

% For positioning figures (eg. images) with [H]
\usepackage{float}

% For styling figure captions
\usepackage{subcaption}
\captionsetup{labelfont=bf, textfont=it, font=footnotesize}

% Image shortcuts
\newcommand{\img}[2]{
  \begin{figure}[H]
    \medskip
    \centering
    \includegraphics[width=1\linewidth]{#1}
    \caption{#2}
    \medskip\label{fig:#1}
  \end{figure}
}
\newcommand{\imgs}[2]{
  \begin{figure}[H]
    \medskip
    \begin{subfigure}{.5\textwidth}
      \centering
      \includegraphics[width=.9\linewidth]{#1-1}
    \end{subfigure}%
    \begin{subfigure}{.5\textwidth}
      \centering
      \includegraphics[width=.9\linewidth]{#1-2}
    \end{subfigure}
    \caption{#2}
    \medskip\label{fig:#1}
  \end{figure}
}

% For styling headers/sections
\usepackage{titlesec}
\titlelabel{\thetitle.\quad}
\titleformat*{\section}{\large\bfseries} % Custom font/size for sections

% Header on each page, and footer
\usepackage{fancyhdr}
\pagestyle{fancy}
\fancyhf{} % sets both header and footer to nothing
\renewcommand{\headrulewidth}{0pt}
\fancyhead[CO]{\footnotesize\textit{Luis F. / Lifecycle Project:
LG 50LB5800}}
\fancyfoot[CO]{\thepage} % page numbers

% Bibliography
\usepackage{csquotes}
\usepackage[style=mla]{biblatex}
\addbibresource{main.bib}

% Links
\usepackage{hyperref}
\hypersetup{colorlinks=true, linkcolor=blue, urlcolor=cyan, citecolor=black}
\urlstyle{same}

% Custom commands
\newcommand\todo[1]{{\color{red}{\footnotesize[TODO:\@ #1]}}}
% show TODO's in red
%\newcommand\todo[1]{} % to remove all todo's

\begin{document}

%\title{\vspace{-3em}\Huge{\textbf{Lifecycle Project: LG 50LB5800}}}
%\title{\vspace{-3em}\Large{\textbf{Lifecycle Project: LG 50LB5800}}}
\title{\vspace{-3em}\Large{\textbf{\mbox{\hspace{-0.25em}Lifecycle
Project: LG 50LB5800}}}} % the mbox and hspace help to fit in one line
\author{\small{Luis F.}}
\date{\vspace{-0.5em}\small{December 2024}}

\maketitle

% Nicer formatting of the abstract enclosed in lines
% \renewenvironment{abstract}
% {
%   \begin{quote}
%   \noindent \rule{\linewidth}{.5pt}\par{\bfseries \abstractname. }
% }{
%   \\ \noindent \rule{\linewidth}{.5pt}\medskip
%   \end{quote}
% }

\vspace{-2em}
% \begin{abstract}\small\itshape
% blah blah blah
% \end{abstract}

% Do I even need a ToC for such a small paper?
\tableofcontents

\section{Criterion A: 3D Model}

For my Digital Societies Lifecycle Project, I decided to research the
lifecycle of an LG 50LB5800 TV.\@ I have one of these at home, so it is
convenient to measure and model. Since I found it being thrown away,
and it currently doesn't work, I'm free to open it up and see inside
if I want to.

\img{lg-50LB5800}{50LB5800 LG Smart TV~\autocite{unknown-author-no-dateB}.}

I decided to model the TV using CAD software, as I had previous
experience using
\href{https://www.autodesk.com/products/fusion-360/overview}{Fusion
360} and thought it was the most appropriate choice for this project.
However, I had issues with licensing and downloading Fusion 360, so I
decided to try out FreeCAD.\@ \href{https://www.freecad.org}{FreeCAD}
is an open-source parametric 3D modeling
software~\autocite{jolahde-2018}. It has a variety of online
resources, like tutorials and documentation, so I was able to learn
it quickly enough for this project.

I started off the model by making a 2D sketch, and making a
rectangle. This would be the base frame of the television. I then
measured the frame of the television using a metre stick, to ensure
the design was of a one-to-one scale. After the frame was done, I
sketched the TV screen. This was more challenging as the only
measurement I had was the diagonal length of the screen being 50
inches, however I was able to make an accurate sketch using constraints.

\img{a1}{Sketching the TV screen in FreeCAD.}

Then I extruded, or in FreeCAD terms ``padded'', the frame. I had to
measure the thickness of the TV to make it to scale.

\img{a2}{Modeling the base of the TV in FreeCAD.}

By taking more measurements and using these basic operations
(sketching and padding, and also a bit of filleting), I was able to
arrive at my final 3D model.

\img{a3}{My final 3D model of a 50LB5800 LG Smart TV.}

I decided to add the LG logo to the back of my model, just like the
real one has. I sketched the logo on top of a reference image of the
logo using regular sketch tools and B-splines for the curves of the G.

\img{a4}{Sketching the LG logo on the back of the 3D model.}

Once I was done modelling, I exported my 3D model as an
\href{https://drive.google.com/file/d/1JmbW4QBFge6CosMKm4P-QzFGuzJ4o9Oy/view?usp=sharing}{STL
file} that slicing software can understand. I used
\href{https://ultimaker.com/software/ultimaker-cura}{UltiMaker Cura}
for slicing my model, as it is what Mr.\ McCallister recommended in
class and is what he showed us the settings for. The settings used
are shown in table~\ref{tab:cura-settings}. The model had to be
scaled down for printing, as the 3D printer is small and this print
was only a prototype anyway. The model was scaled in Cura to 15\% of
its original size.

\img{a5-cura}{Scaling and slicing the 3D model in UltiMaker Cura.}

\begin{table}[H]
  \centering
  \begin{tabular}{ccc}
    Setting & Default Value & New value \\
    \hline
    Infill Density & 20.0\% & 15.0\% \\
    Generate Supports & False & True \\
    Support Structure & Normal & Tree \\
    Printing Temperature & 200.0 \textdegree C & 215.0 \textdegree C \\
  \end{tabular}
  \caption{Settings used for slicing the model in UltiMaker
  Cura.}\label{tab:cura-settings}
\end{table}

Once the model was sliced into G-code, I saved the G-code onto an SD
card and put it in one of the printers. After cleaning the print
plate and applying some glue, I set it off to print. The first two
prints failed, so I tried a different 3D printer and was able to get
it to print well enough.

\todo{Image of the printed 3D model.}

\section{Criterion B: Product Production Inquiry}

\subsection{Production Questions and Addressing Strategies}

\begin{itemize}
  \item How does/did LG make the IPS LED displays for its LCD TVs?
    \begin{itemize}
      \item Research LG LCD production process and the role of LG
        Display and LG Chem.
    \end{itemize}

  \item What are the environmental impacts of LG's TV production
    process?
    \begin{itemize}
      \item Research LG's sustainability initiatives and the
        environmental impacts of TV production.
    \end{itemize}

  \item What are the ethical concerns with LG's TV production?
    \begin{itemize}
      \item Research LG's conflict mineral policy and the ethical
        concerns with TV production.
    \end{itemize}

  \item Where are the raw materials for LG's TVs sourced from?
    \begin{itemize}
      \item Research LG's supply chain and raw material sourcing, and
        maybe the role of LG Electronics and LG Chem.
    \end{itemize}
\end{itemize}

\subsection{General Production Process}

This specific model of the 50LB5800 LG Smart TV that I have at home
was assembled in Poland, July 2014, as can be seen on the
informational sticker on the back of the TV in
figure~\ref{fig:lg-back-label}. The TV was likely produced in the LG
production plant in Mława, Poland~\autocite{lg-2020}. The LG Display
plant in Mława, however, was relocated to Wrocław in 2016 and is no
longer operated by LG
Display~\autocite{allen-2016}~\autocite{evertiq-ab-2016}, being sold
to LG Chem instead~\autocite{shah-2024}. The TV was likely produced
in the Mława plant before it was relocated. This model of the TV is
also now discontinued and no longer produced~\autocite{unknown-author-no-dateB}.

\img{lg-back-label}{The informational sticker on the back of my LG
50LB5800 Smart TV.}

\img{lgensol-wroclaw-google-maps}{The LG plant in Wrocław, Poland.
Image from \href{https://maps.app.goo.gl/vMukYzKtPBX12RGo9}{Google Maps}.}

The production process of the LG 50LB5800 Smart TV is not
specifically documented, but it can be inferred from the general
production process of LG TVs. Jeremy Kaplan, a former editor-in-chief of
Digital Trends, visited the LG Display facility in Gumi, South Korea,
and described the manufacturing process of LG's TVs.

\begin{quote}
  Because technology changes so rapidly, assembly lines are designed
  to be modular. They look impermanent, almost transient, the sort of
  thing you would put together yourself if someone told you to make one.

  Conveyor belts rolled along as straight as an arrow the entire
  length of the room, perhaps a quarter mile. Screens came down from
  the ceiling to workstations where arms, either robotic or human,
  attached the few circuit boards required to transmit and process the picture.
\end{quote}

\imgs{gumi-facility}{The LG Display facility in Gumi, South
Korea~\autocite{kaplan-2016}.}

\todo{Is this enough for criterion B?}

\subsection{Ethical Concerns}

As with any electronic device, there are ethical concerns with the
production of LG TVs. The TVs will contain conflict minerals, like
tin, tantalum, tungsten, and gold. These minerals are sourced from
conflict regions, like the Democratic
Republic of the Congo, and are often mined using child labor and
under dangerous conditions~\autocite{hower-2013}. LG has a policy to
avoid using conflict
minerals, but it is difficult to ensure that all minerals are
conflict-free~\autocite{lg-electronics-no-date}. In LG's Policy for
Conflict Materials, LG states the following:

\begin{quote}
  LGE is committed to adopting, widely disseminating and incorporating
  principles in support of these goals in contracts, agreements and/or
  communications with suppliers. LGE expects our suppliers to have in
  place policies and due-diligence measures to facilitate the sourcing
of minerals that are ``DRC conflict free.\textsuperscript{3)}'' In
addition, LGE requires
our suppliers to comply with LGE's Supplier Code of Conduct, based on
the Responsible Business Alliance (RBA) Code of Conduct, which sets
forth LGE's broader standards for suppliers and includes provisions
relating to human rights, ethical conduct, and environmental
protection as well as additional provisions relating to conflict minerals.
\end{quote}

LG, like most big companies, officially promotes ethically and
environmentally good practices. However, often the reality doesn't
live up to their public claims. For instance, in 2011, the British
NGO Friends of the Earth released a report that showed the
devastation caused by mining for tin on Bangka island in Indonesia.
Around a third of the world's mined tin comes from Bangka and
neighboring island Belitung. At the time LG said: ``We can confirm
that we do not directly source any products from Bangka, but our
investigations have revealed that some of the tin used by our
third-party suppliers may come from this region''. So the company
was, at this time, still involved in using tin that was being sourced
unsustainably and unethically~\autocite{hower-2013}.

\subsection{Raw Materials}

As of 2023, LG has 249 suppliers of conflict minerals in 41
countries. Only two of these are in Poland, both conforming to LG's
audit protocols. One of them is Fenix Metals, a supplier of tin, and
the other is KGHM Polska Mied\'z, a supplier of
gold~\autocite{lg-electronics-2023}. Tin is used for soldering
electronic and metallic components, and gold is used for circuit
board connectors~\autocite{brigham-2023}. An example of where these
minerals are used in the LG 50LB5800 Smart TV is in its mainboard, as
seen in figure~\ref{fig:EBT62999602}, which contains gold connectors
and tin soldering.

\subsubsection{Fenix Metals: Tin Supplier}

Fenix Metals is a tin recycling company in Poland, that manufactures
``quality tin and alloys from recycled tin
residues''~\autocite{fenix-metals-no-dateB}. Fenix Metals is a
supplier of tin to LG, which is used in the soldering of electronic
and metallic components in LG TVs. Fenix Metals don't source their
tin from conflict regions, and are a member of the International Tin
Association~\autocite{international-tin-association-2022}, which
promotes responsible and sustainable tin
production~\autocite{international-tin-association-2024}.

\imgs{fenix-metals}{Fenix Metals' tin ingots and other
products~\autocite{fenix-metals-no-dateC}.}

Fenix Metals accepts tin residues from its suppliers, which are then
processed and refined into tin and tin alloys. The company does not
state the process used to process and refine the tin, but it is likely that they
use a smelting process to extract the tin from the residues. The
smelting process involves heating the tin residues to high temperatures
to melt the tin, which then produces a pool of molten tin and scoria
containing impurities. The molten tin is then made into large slabs that can be
further refined, and the scoria is retreated to extract any remaining
tin~\autocite{barry-1999}.

There are several methods for refining tin, such as fire refining and
electrolytic refining. Fire refining involves heating the tin to high
temperatures to remove impurities, while electrolytic refining uses
electricity to remove impurities from the tin. Fire refining tends to be
most commonly used and produces up to 99.85\% pure tin, while
on the other hand, electrolytic refining produces up to 99.999\% pure
tin~\autocite{barry-1999}.

An example of where Fenix Metals' tin is used in the LG 50LB5800 Smart
TV is in the soldering of the PCB, as seen in
figure~\ref{fig:EBT62999602}. The mainboard also goes through a
surface treatment process called immersion tin. Immersion tin involves
replacing copper with tin on
PCB solder pads, and has several advantages and disadvantages. Some
advantages include increased solderability, versatility, and
environmental friendliness, while some disadvantages include quicker
oxidation and soldering reliability issues~\autocite{jenell-2023}.

\img{EBT62999602}{The LG 50LB5800 Smart TV
mainboard~\autocite{tv-parts-canada-2024}.}

\subsubsection{KGHM Polska Mied\'z: Gold Supplier}

Gold is used in several parts of the TV, such as the PCB, connectors
and microchips. Gold is used in gold plating for the PCB for
efficient and reliable electrical connections, in connectors for its
low resistance and good durability, and in microchips for its high
conductivity~\autocite{elmore-2024}.

\imgs{kghm-gold}{KGHM Polska Mied\'z's gold
bars~\autocite{kghm-polska-miedz-no-dateB}.}

KGHM exports gold bars to LG, which are used in the production of
circuit board connectors in LG TVs. The bars weigh one of 0.5kg, 1kg,
4kg, 6kg or 12kg, and are guaranteed to have no more than 100ppm of
impurities. The gold bars are produced from anode slime, a byproduct
of the electrolytic silver refining process. The anode slime is
processed in hydrometallurgical processes to extract the gold, which
is then melted in an induction furnace and cast into gold bars. The
gold bars are then exported to LG for use in the production of
circuit board connectors~\autocite{kghm-polska-miedz-no-dateB}.

\section{Criterion C: Product Map}

The LG 50LB5800 Smart TV goes through several stages in its lifecycle,
from raw material extraction, to production, to distribution, to
consumption, to disposal. The map in figure~\ref{fig:lg-map} shows the
relevant locations in the lifecycle of the LG 50LB5800 Smart TV.\@

The map was created using Google My Maps, a web mapping service that
allows users to create custom maps with markers, lines, and shapes.
Interactive versions of the map can be found online on
\href{https://www.google.com/maps/d/edit?mid=1_aeai9S8RSvkTi9m7rJfM217r6kOGgc&usp=sharing}{Google
Maps} and on
\href{https://www.google.com/maps/d/earth?mid=1_aeai9S8RSvkTi9m7rJfM217r6kOGgc}{Google
Earth}.

\img{lg-map}{Map of relevant locations in the lifecycle of the LG
50LB5800 Smart TV.}

\section{Criterion D: Product Lifespan, Disposal, and Impact}

The television that I found was made in July 2014. Our neighbours
discarded it this year. It has lasted 10 years. On average, modern
televisions last 10 years, so it has lasted the average length of
time. According to ``Smart TV Club'', LG televisions last as long as
Samsung products, but not as long as Sony ones, which can potentially
last a year or two longer~\autocite{raj-2024}.

This television switches on, so it seems that the problem is a faulty
connection, a copper wire or the tin solder.

The internal components within a TV play a big role in its lifespan.
Better quality components typically lead to better heat dissipation
and durability. Manufacturers may want to save on production costs by
using cheaper components, thereby increasing profits. Although this
is good for their profits, it's detrimental to the environment, as
more resources will be needed to make more new televisions to replace
the ones that break quicker.

There are, however, ways to extend the lifespan of a TV.\@ One way is to
keep it clean and dust-free. Dust can accumulate on the internal
components of a TV, causing them to overheat and fail. Another way is
to keep the TV in a well-ventilated area. TVs need to be kept cool to
prevent overheating. Overheating can cause the internal components to
fail. A third way is to avoid power surges. Power surges can damage
the internal components of a TV, causing them to fail. A surge
protector can be used to protect the TV from power surges. Software
updates can also help extend the lifespan of a TV.\@ Software updates
can fix bugs and security vulnerabilities, preventing the TV from
malfunctioning or being hacked~\autocite{raj-2024}.

\printbibliography[heading=bibintoc,title=References]

\end{document}

